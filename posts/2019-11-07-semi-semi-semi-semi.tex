\documentclass{article}
%include polycode.fmt
%
\begin{document}

Semigroups, semigroupoids... are there more semithings?

\subsection{Group, Semigroup, Monoid}

Let's start "from the beginning". Groups is 19th century thing.  A \emph{group}
is a set $G$ with a binary operation $\bullet$, identity element $e$ and
inverse element $a^{-1}$ for each $a \in G$.
With an obvious laws you can imagine relating these.

If we remove \emph{identity} element from a group, "obviously" we get a \emph{semigroup}.
Because if there's \emph{any} element $x \in G$, we can recover the identity
element by $e = x \bullet x^{-1}$.

So what's a semigroup with identity element? For sometimes it were just that,
until we started call it \emph{monoid}. Go figure, naming is hard, not only in
programming, but also in mathematics.\foonote{\url{https://math.stackexchange.com/questions/156952/why-the-terminology-monoid}}

\subsection{Category, Groupoid, Semigroupoid}

You are hopefully familiar with a concept of a category.
I repeat a definition here:

\begin{definition}[Awodey 1.1]\textbf{Definition} (Awodey 1.1)
A \emph{category} consist of the following data
\begin{itemize}
\item Objects: $A, B, C, \ldots$
\item Arrows: $f, g, h, \ldots$

\item For each arrow $f$, there are given objects

\begin{equation*}
\mathrm{dom}(f), \qquad \mathrm{cod}(f)
\end{equation*}

called the \emph{domain} and \emph{codomain} of $f$. We write

\begin{equation*}
f : A \to B
\end{equation*}

to indicate that $A = \mathrm{dom}(f)$ and $B = \mathrm{cod}(f)$.

\item Given arrows $f : A \to B$ and $g : B \to C$, that is, with

\begin{equation*}
\mathrm{cod}(f) = \mathrm{dom}(g)
\end{equation*}

there is given an arrow

\begin{equation*}
g \circ f : A \to C
\end{equation*}

called the \emph{composite} of $f$ and $g$.

\item For each object $A$, there is given an arrow

\begin{equation*}
1_A : A \to A
\end{equation*}

called the \emph{identity arrow} of $A$.

\item Associativity:

\begin{equation*}
h \circ (g \circ f) = (h \circ g) \circ f
\end{equation*}

for all $f : A \to B$, $g : B \to C$, $h : C \to D$.

\item Unit:

\begin{equation*}
f \circ 1_A = f = 1_B \circ f
\end{equation*}

for all $f : A \to B$.
\end{itemize}
\end{definition}

If you think hard (or read a book), you'll learn that a single object
category is a monoid: category arrows are monoid elements, and the
laws work out.

The group analogue is called \emph{groupoid}. In addition to
category data, we require that for each arrow $f : A \to B$ there is an inverse arrow $f^{-1} : B \to A$,
such that $f \circ f^{-1} = 1_B$ and $f^{-1} \circ f = 1_A$.
Or more succinctly: that each arrow is an isomorphism.

But we can also remove stuff: if we remove identity arrows,
and unit law we get \emph{semigroupoid}.

\begin{definition}\textbf{Definition}
A \emph{semigroupoid} consist of the following data
\begin{itemize}
\item Objects: $A, B, C, \ldots$
\item Arrows: $f, g, h, \ldots$

\item For each arrow $f$, there are given objects

\begin{equation*}
\mathrm{dom}(f), \qquad \mathrm{cod}(f)
\end{equation*}

called the \emph{domain} and \emph{codomain} of $f$. We write

\begin{equation*}
f : A \to B
\end{equation*}

to indicate that $A = \mathrm{dom}(f)$ and $B = \mathrm{cod}(f)$.

\item Given arrows $f : A \to B$ and $g : B \to C$, that is, with

\begin{equation*}
\mathrm{cod}(f) = \mathrm{dom}(g)
\end{equation*}

there is given an arrow

\begin{equation*}
g \circ f : A \to C
\end{equation*}

called the \emph{composite} of $f$ and $g$.

\item Associativity:

\begin{equation*}
h \circ (g \circ f) = (h \circ g) \circ f
\end{equation*}

for all $f : A \to B$, $g : B \to C$, $h : C \to D$.

\end{itemize}
\end{definition}

A reader probably ask themselves: are there interesting, not-contrived
examples of semigroupoids, which aren't also categories?
There are. If poset (set with partial order) is 
an example of category, then a set with strict order\footnote{\url{http://mathworld.wolfram.com/StrictOrder.html}, I not dare to call the set with strict order a soset},
is an example of semigroupoid.

As a concrete example, natural numbers with an unique arrow between $n$ and $m$ when $n < m$ is a semigroupoid.

\begin{equation}
0 < 1 < 2 < 3 < 4 < \cdots
\end{equation}

There are no identity arrow, as $n \not< n$, but associativity works out:
if $n < m$ and $m < p$ then $n < p$. Let's call this semigroupoid $\mathbf{LT}$.

Finally, a plot twist. \href{https://ncatlab.org/nlab/show/semicategory}{nLab} calls semigroupoids semicategories,
and don't even mention a semigroupoid as an alternative name!

\subsection{Functors, Semifunctors...}

Recall a definition of functor.

\begin{definition}[Awodey 1.2]\textbf{Definition} (Awodey 1.2)
A \emph{functor}

\begin{equation*}
F : \mathbf{C} \to \mathbf{D}
\end{equation*}

between categories $\mathbf{C}$ and $\mathbf{D}$ is a mapping of objects
to objects and arrows to arrows, in such a way that
\begin{itemize}
\item $F (f : A \to B) = F(f) : F(A) \to F(B)$,
\item $F(1_A) = 1_{F(A)}$,
\item $F(g \circ f) = F(g) \circ F(f)$.
\end{itemize}
That is, $F$ preserves domains and codomains, identity arrows,
and composition. A functor $F : \mathbf{C} \to \mathbf{D}$ thus gives
a sort of "picture" -- perhaps distorted -- of $\mathbf{C}$ in $\mathbf{D}$.
\end{definition}

Functors preserve identities, but semigroupoids don't have identities to be preserved.
We need a weaker concept:

\begin{definition}\textbf{Definition}
A \emph{semifunctor}

\begin{equation*}
F : \mathbf{C} \to \mathbf{D}
\end{equation*}

between semigroupoids $\mathbf{C}$ and $\mathbf{D}$ is a mapping of objects
to objects and arrows to arrows, in such a way that
\begin{itemize}
\item $F (f : A \to B) = F(f) : F(A) \to F(B)$,
\item $F(g \circ f) = F(g) \circ F(f)$.
\end{itemize}
That is, $F$ preserves domains and codomains,
and composition. A functor $F : \mathbf{C} \to \mathbf{D}$ thus gives
a sort of "picture" -- perhaps distorted -- of $\mathbf{C}$ in $\mathbf{D}$.
\end{definition}

An identity functor is obviously a semifunctor, also a successor functor $S : \mathbf{N} \to \mathbf{N}$
is a semifunctor $\mathbf{LT} \to \mathbf{LT}$.

In Haskell, that would be silly to define a class for (endo)semifunctors:

\begin{code}
-- semimap f . semimap g = semimap (f . g)
class Semifunctor f where
    semimap :: (a -> b) -> f a -> f b
\end{code}

It's the |Functor| type-class without an identity law.
On the other hand, something like

\begin{code}
data Mag a b t where
    Map   :: (x -> t) -> Mag a b x -> Mag a b t
    One   :: a -> Mag a b b

instance Semifunctor (Mag a b) where
    fmap f (Map g x) = Map (f . g) x
    fmap f x         = Map f x
\end{code}

would be valid.

\subsection{Semimonad?}

Now as we have semifunctors, would it make sense to ask whether
endosemifunctors can form a monad

\begin{definition}\textbf{Semimonad}
A \emph{semimonad} on semigroupoid $C$ consists of
\begin{itemize}
\item an endosemifunctor $T : C \to C$
\item semi natural transformation $\eta : 1_C \to T$ (|return|)
\item semi natural transformation $\mu : T^2 \to T$  (|join|)
\item Associativity (as semi natural transformations $T^3 \to T$)

\begin{equation}
\mu \circ T \mu = \mu \circ \mu T
\end{equation}

\item Identity (as semi natural transformations  $T \to T$)

\begin{equation}
\mu \circ T\eta = \mu \circ \eta T = 1_T
\end{equation}
\end{itemize}
\end{definition}

Looks a lot like monad, but for semifunctors. I have an example: the $S$ semifunctor
is a semimonad. Feels like (I didn't check) that all strictly monotonic functions would fit.
We need to find some more structured semigroupoid than $\mathbf{LT}$ to find more interesting semimonads, but I haven't yet.

I end with a catchy phrase:

\emph{A semimonad is a monoid (!) in the category of endosemifunctors.}

What would be a semigroup in the category of endofunctors\footnote{in Haskell we call it |Bind|, at least now (or |Apply| for different semigroup)}, or
a semigroup in the category of endosemifunctors?

Naming is hard.

\subsection{References: More on the theory of semi-functors}

There's a paper \emph{The theory of semi-functors} by
Raymond Hoofman \url{https://doi.org/10.1017/S096012950000013X}.

\end{document}
