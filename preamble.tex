\documentclass[10pt,a4paper,pdftex]{article}
\usepackage[utf8]{inputenc}
\usepackage[T1]{fontenc}
%\usepackage{fullpage}
\usepackage{centernot}
\usepackage{amsmath}
\usepackage{amssymb,amsthm}
\usepackage{mathrsfs} % mathscr
\usepackage{mathtools} % coloneq
\usepackage[makeroom]{cancel} % cancel
\usepackage{color}
\usepackage[a4paper,inner=3cm,outer=3cm,top=3cm,bottom=3cm,marginparsep=5mm,marginparwidth=3cm]{geometry}
\usepackage{microtype}
\usepackage{float}
\usepackage[english]{babel}

% footnotes
\interfootnotelinepenalty=10000

% agdaid
\usepackage{xstring}
\newcommand{\agdaid}[1]{\mathit{#1}}

% TODO:
\usepackage{prftree}
% ftp://ftp.funet.fi/pub/TeX/CTAN/macros/latex/contrib/prftree/prftreedoc.pdf
\setlength{\prfinterspace}{1.5em}
\setlength{\prfrulenameskip}{0.3em}
\setlength{\prflinethickness}{.4pt}

% Verbatim
\usepackage{fancyvrb}

% Landscape
\usepackage{lscape}

% Colors
\usepackage{xcolor}
\definecolor{darkgreen}{rgb}{0,.5,0}
\definecolor{darkblue}{rgb}{0,0,.5}

% Fonts
\usepackage{mathpazo}
\usepackage{helvet}
%\usepackage{courier}
\usepackage{inconsolata}
\usepackage{microtype}

% graphics
\usepackage{graphics}
\DeclareGraphicsRule{*}{mps}{*}{}

% tables
\usepackage{booktabs}

% braces
\usepackage{stmaryrd}

% bibliography
\usepackage[square,sort&compress,numbers]{natbib}
\bibliographystyle{acm}
\renewcommand*{\bibfont}{\footnotesize}

% lhs2TeX
% \usepackage{poly}

% enumerations
\usepackage{enumitem}

% stuff
\usepackage{calc}

% Unicode
\usepackage{newunicodechar}
\newunicodechar{ℕ}{\ensuremath{\mathbb{N}}}
\newunicodechar{∀}{\ensuremath{\mathbb{\forall}}}
\newunicodechar{′}{'}
\newunicodechar{∷}{::}
\newunicodechar{≡}{\ensuremath{\equiv}}
\newunicodechar{→}{\ensuremath{\to}}
\newunicodechar{⇒}{\ensuremath{\Rightarrow}}
\newunicodechar{≤}{\ensuremath{\le}}
\newunicodechar{⊔}{\ensuremath{\sqcup}}
\newunicodechar{⊥}{\ensuremath{\bot}}
\newunicodechar{₁}{\ensuremath{{}_1}}
\newunicodechar{₂}{\ensuremath{{}_2}}
\newunicodechar{λ}{\ensuremath{\lambda}}
\newunicodechar{ℓ}{\ensuremath{\ell}}
\newunicodechar{Γ}{\ensuremath{\Gamma}}
\newunicodechar{Δ}{\ensuremath{\Delta}}
\newunicodechar{∈}{\ensuremath{\in}}
\newunicodechar{∘}{\ensuremath{\circ}}
\newunicodechar{ε}{\ensuremath{\varepsilon}}
\newunicodechar{τ}{\ensuremath{\tau}}
\newunicodechar{σ}{\ensuremath{\sigma}}
\newunicodechar{Σ}{\ensuremath{\Sigma}}
\newunicodechar{⊤}{\ensuremath{\top}}
\newunicodechar{⋆}{\ensuremath{\star}}

% finnish unicode :P
\newunicodechar{Ö}{\"O}
\newunicodechar{Ä}{\"A}

% References
\usepackage[hyphens]{url}
\usepackage[pdftex]{hyperref}
\hypersetup{pdfborder={0 0 0}}
\hypersetup{bookmarksnumbered=true}
% The following line suggests the PDF reader that it should show the
% first level of bookmarks opened in the hierarchical bookmark view.
\hypersetup{bookmarksopen=true,bookmarksopenlevel=1}
% Hyperref can also set up the PDF metadata fields. These are
% set a bit later on, after the thesis setup.
\hypersetup{colorlinks=true,linkcolor=darkgreen,urlcolor=darkblue}
\usepackage[capitalize,nameinlink,noabbrev]{cleveref}

% Theorems
% http://en.wikibooks.org/wiki/LaTeX/Theorems

% http://tex.stackexchange.com/questions/46258/how-to-get-correct-autoref-for-theorems
%\theoremstyle{definition}
%\newtheorem{definition}{Definition}[chapter]
%\newtheorem{example}[definition]{Example}
%\newtheorem{remark}[definition]{Remark}

%\newcommand{\definitionautorefname}{Definition}
%\newcommand{\exampleautorefname}{Definition}

% Currently the English versions are used for the PDF file metadata
% Set the PDF title
%\hypersetup{pdftitle={\TITLE\ \SUBTITLE}}
% Set the PDF author
%\hypersetup{pdfauthor={\AUTHOR}}
% Set the PDF keywords
%\hypersetup{pdfkeywords={\KEYWORDS}}
% Set the PDF subject
%\hypersetup{pdfsubject={Master's Thesis}}

% Geometry

% Line height
%\setlength{\baselineskip}{20mm}
% \setlength{\linespread}{1.5}

% paragraph style
\setlength{\parindent}{0em}
\setlength{\parskip}{0.3em}

% pandoc
\providecommand{\tightlist}{%
  \setlength{\itemsep}{0pt}\setlength{\parskip}{0pt}}
